\documentclass{article}
\usepackage{amsmath}
\usepackage[margin=0.5in]{geometry}
\renewcommand{\baselinestretch}{1.5}
\usepackage{mathtools}
\author{Trixie Roque}
\title{Homework 2}
\date{03/02/15}
\begin{document}
\maketitle

\begin{enumerate}
	\item (a)$ f = \Theta (g) \\
		(b) f = O (g)\\
		(c) f = \Theta (g)\\
		(d) f = \Theta (g)\\
		(e) f = \Theta (g)\\
		(f) f = \Theta (g)\\
		(g) f = O (g)\\
		(h) f = \Omega (g)\\
		(i) f = O (g)\\
		(j) f = O (g)\\
		(k) f = \Omega (g)\\
		(l) f = O (g)\\
		(m) f = O (g)\\
		(n) f = \Theta (g)\\
		(o) f = \Omega (g)\\
		(p) f = O (g)\\
		(q) f = \Theta (g)\\
		$
		
	\item (a) \\ $$ \begin{pmatrix}  
				a & b \\ c & d
			   \end{pmatrix}
			   \begin{pmatrix}
			   	e & f \\ g & h
			   \end{pmatrix}
			   =
			   \begin{pmatrix}
			   	(a)(e) + (b)(g) & (a)(f)+(b)(h) \\ (c)(e) + (d)(g) & (c)(f)+(d)(h)
			   \end{pmatrix}
			   =
			   \begin{pmatrix}
			   	ae + bg & af+bh \\ ce + dg & cf+dh
			   \end{pmatrix}
			$$
			\\
		(b) \text{It takes  f = O ($\log(n)$) to compute $X^n$ since we can just use the method of repeatedly squaring $X^1$} \\ 
		\text{and finding powers of 2 such that when multiplied together, the exponents equal n. For example, consider $X^8$.} \\
		\text{Then $X^8$ = $X^{1*2*2*2*2}$ = $X^{2*2*2}$ = $X^{4*2}$ = $X^8$}. Then we can see that it is faster to just double the exponents each time (which is the same as multiplying a matrix by itself) in order to reach a power of 8 instead of multiplying the matrix $X^1$ 8 times. Using the repeated squaring method for $X^8$ took only 3 multiplications compared to 8 which is $\log_2(8)$ times faster. Therefore, for $X^n$, it suffices to have $\log_2(n)$ matrix multiplications.
		
	\item \ x = number of bits\\
		$  \frac{x\log_2(2)} {\log_2(10)}\ = \frac{x}{0.301}\ = x*3.3 < 4 * x \\
		$ \ so a decimal with x digits will have, at most, 4 times as many digits when converted to 	binary. The ratio of digits between binary and decimal numbers is 
		$ \frac{\log_2(2)} {\log_2(10)}\ $
		\ or about 3.3
		
	\item \ We first show the upper bound: \ n! = $n^n$ \\
		$ \log(n!) = \log(1) + \log(2) + \log(3) + . . . + \log(n)
		  \leq \log(n) + \log(n) + \log(n) + . . . + \log(n) \\
		 $ \text We replaced $\log(1) + . . . + \log(n-1)$ with $\log(n)$ which guarantees that $\log(n) + . . . + \log(n)$ is greater than $\log(1) + ... + \log(n).$
		 $
		 $ \text Since the number of $\log(n)$'s is n, the upper bound is O(n$\log(n)$). $ \\ \\
		 $ \text We then show the lower bound using the same method: $ \ n! = $ $\frac{n}{2}^\frac{n}{2} $ $ \\
		$ $ \log(n!) = \log(1) + \log(2) + . . . + \log(\frac{n}{2}) + . . . + \log(n) $
		 $ \geq \log(\frac{n}{2}) + . . . + \log(n) \geq \log(\frac{n}{2}) + . . . + \log(\frac{n}{2}) \\
		 $ \text Here we truncate the first half of $\log(1) + . . . + \log(n)$ which guarantees that $\log(\frac{n}{2}) + . . . + \log(n)$ is less than $\log(1) + ... + \log(n)$. Then since there are only half as many $\log(\frac{n}{2})$, the number of $\log(\frac{n}{2})$ we have is $\frac{n}{2}$. Therefore, the lower bound is $\frac{n}{2}\log(\frac{n}{2})$ or O($n\log(n)$) $
		 $
		 $
		$ \\
		
		
	\item $ \text {Yes (ran in Python)} $\\
		($4^{1536}$ - $9^{4824}$) mod 35 = [($4^{1024}$*$4^{512}$) - ($9^{4096}$ * $9^{512}$ * $9^{128}$ * $9^{64}$ *$9^{16}$ * $9^8$)]mod 35 = 0
		\\
	
	\item $ \text {Yes (ran in Python)} $\\
		($5^{30000}$ - $6^{123456}$) mod 31 = 0 \\
	
    	\item $ \text{Consider b = 15. Then $a^{15}$ = $a^7$ * $a^7$ * $a^1$. Then this method only uses 3 multiplications where the} \\ $
	$\text {exponents = 7 + 7 + 1 = 15. Compared to the repeated squaring method which uses 4 multiplications,} \\ $
	$\text{ $a^1$ * $a^2$ * $a^4$ * $a^8$, this other method finds the result with less calculations.}
		$
		\\
	
	\item $2^{125}$ mod 127 = ($2^{64}$ * $2^{32}$ * $2^{16}$ * $2^{8}$ * $2^{2}$ * $2^{2}$ * $2^{1}$) mod 127 \\
		$2^{125}$ mod 127 = (((((($2^{64}$ * $2^{32}$) mod 127 * $2^{16}$) mod 127 * $2^{8}$) mod 127 * $2^{2}$) mod 127 * $2^{2}$) mod 127 * $2^{1}$) mod 127
		$2^{125}$ mod 127 = 64 (method used: plugged into Python) 
		\\
	
	\item $ \text {See Github repo: Homework 2 "lcm.py"} $
		\\
	
	\item $ \text{Idea taken from Wikipedia on Wilson's theorem:} \\
		$ \text{Since Wilson's theorem uses factorials, the bigger n gets, the more complex the computation} \\ 
		$ \text{will be thus the computation will have a longer running time.}$
		\\
	
	\item $ \text {See Github repo: Homework 2 "exponential$\_$mod.py"} $
		\\
	
\end{enumerate}



\end{document}